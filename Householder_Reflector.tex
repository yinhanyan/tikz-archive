\documentclass{standalone}
% \documentclass[dvisvgm]{minimal}
% \special{background White}


\usepackage{bm}
\usepackage{tikz}
\usetikzlibrary{calc} 
\usetikzlibrary{angles}

\begin{document}
% \begin{figure}
%   \centering

\begin{tikzpicture}
  \draw[->] (-2,0) -- (6,0) node[anchor=north] {$x$};
  \draw[->] (0,-1) -- (0,6) node[anchor=east] {$y$};
  \draw[->, thick] (0,0) -- (2,4) coordinate(x) node[anchor=south] {$\bm{x}$};
  \draw[dashed] (-2,-1) coordinate(p1) --   (8, 4) coordinate (p2) node[right]{range($\bm{P}$)};

  \draw[dashed] (x) -- ($(p1)!(x)!(p2)$) coordinate(px);
  \draw[->, thick] (0,0) -- (px) node[anchor=north] {$\bm{Px}$};
  \coordinate (d_px) at ($2*(px)$);
  \draw[->, thick] (0,0) -- (d_px) node[anchor=south] {$2\bm{Px}$};
  \draw[->, very thick] (d_px) -- node[anchor = south]{$(\bm{I}-2\bm{P})\bm{x}$} (x);
  \coordinate (O) at (0,0);
  \draw (px) pic [draw,angle radius=4mm] {right angle = x--px--O};
  \draw (O) pic [draw, angle radius=4mm]{angle = px--O--x};
  \draw (d_px) pic [draw, angle radius=4mm]{angle = x--d_px--px};
\end{tikzpicture}

%   \caption{Geometric interpretation of $\bm{I}-2{P}$. }
%   \label{fig:6}
% \end{figure}
\end{document}